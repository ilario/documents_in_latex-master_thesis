We synthesized three chiral polythiophenes using two different enantiomeric excesses and two different polymerization procedures, exploiting the living chain-growth polymerization of polythiophenes. 
Our polythiophenes are unreported in the literature.

From {\HNMR}, {\CNMR} and \gls{NMR2D} we verified a high regioregularity (see Section~\ref{sec:regioregolarita}). We observed (via \gls{SEC} analysis) that the addition of \ch{LiCl} to the polymerization resulted in a higher polymerization degree but increased also the polydispersity giving rise to a small amount of dimerization of the growing chain (see Section~\ref{sec:weight}). The higher weight was confirmed by \gls{UVvis} and \gls{FTIR} spectroscopies and by \gls{SEC}\-/\gls{MALDI}. Furthermore from \gls{MALDI} we saw that our polymers have various terminating groups, conversely to the results reported in the literature for well-defined polythiophenes (see Section~\ref{sec:sintesi-terminazione}). 

The chiral center, incorporated in the side groups, allowed us to perform a chiroptical study on the polymers in solution, in thin film and in solid powder state. The aggregation behavior in good-poor solvent mixtures and the supramolecular structure of these polymers were studied by means of \gls{UVvis} absorption, emission (\gls{PL}) and \acrfull{CD} spectroscopy. From \gls{PL} in good-poor solvents we can clearly notice differences between enantiopure and enantioenriched polymers (see Section~\ref{sec:pl}). 
The \gls{CD} results indicate the presence of more than one aggregation morphology, giving rise to similar \gls{CD} signals. 
Measuring the \gls{CD} intensity at various polymer concentrations we noticed a small nonlinearity which could be related to a multi\-chain event (see page~\pageref{nonlinearita}). 
Comparing the circular dichroism from enantiopure and from enantioenriched polymers we have verified that majority-rules effect is not present or it's too weak to be observed in our data (see page~\pageref{majority}). The solid state structure was investigated by means of \gls{XRD} and \acrshort{XRD2D} showing the presence of both amorphous domains and crystallized domains with the characteristic distances of an interdigitated regioregular poly\-(3-alkyl\-thio\-phene) (see Section~\ref{xrd}). No long range repeating structure was observed using \acrshort{SAXS}. Hence we tend to exclude the presence of large crystalline domains with a small tilt between lamellae giving rise to circular dichroism. More likely there is some extent of chiral structuring in amorphous regions. 

The longest enantiopure polymer was employed in the active layer of a solar cell as donor material in blend with \gls{PCBM}. Preliminary results show an open-circuit voltage $V_{oc}$ higher than for the standard \gls{P3HT}:\-\gls{PCBM} blend (see Section~\ref{cella}). This is caused by the electron withdrawing effect of the oxygen in the side chain of our polymer. 

Starting from the halogenated chain terminations we followed a novel convenient strategy to functionalize our higher weight enantiopure polythiophene \cmpd+{ig2-8} with \gls{BrPhEtTIPNO}, a \acrfull{NMRP} initiator and mediator. Then a short \acrfull{P4VP} block was polymerized obtaining an amphiphilic diblock copolymer (see Section~\ref{sec:nmrp}). {\HNMR} characterization of properly purified material confirms the formation of a block copolymer and allows us to determine the amount of inserted 4-vinyl\-pyridine. 
In spite of the smallness of the \gls{P4VP}, an effect can be observed in the preliminary \gls{CD} studies on aggregation from good-poor solvent mixtures. The coordinating property of \gls{P4VP} block with respect to \gls{PCBM} or semi\-conducting nano\-crystals can allow one to obtain a bi\-continuous pathway in a stable nano\-segregated active layer of a polymeric solar cell. Thus the present approach can unlock the possibility to characterize by circular dichroism spectroscopy the active layer of a stable bulk heterojunction solar cell. 
To the best of our knowledge we present here the first ``chiral thiophene-non thiophene'' copolymer. 